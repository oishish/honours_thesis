%% The following is a directive for TeXShop to indicate the main file
%%!TEX root = diss.tex

\chapter{Abstract}

We have tested whether the entropy of a quantum system can be measured nonlocally, using a capacitively coupled quantum dot as a sensor. We have extended upon the research already done in mesoscopic circuits showing that the entropy of the first few electron ground states in a quantum dot can be locally measured directly. To demonstrate whether this local measurement can be extended to measure the entropy of a nonlocal quantum system, we have probe a simple two state system made up of an electron in a superposition between a quantum dot and a reservoir. Entropy measurements were completed by capacitively coupling this two-state system to a probe quantum dot whose occupation affects the degeneracy of the two-state system and by extension, its entropy. In the dot acting as the probe dot, change in entropy of the entire system was measured by measuring shifts in the occupancy transition from 0 to 1 electrons as a function of temperature. By showing that we can measure the entropy of a quantum system nonlocally — or by measuring only the properties of a nearby, coupled quantum dot — we provide a path to distinguish more novel entangled states with unique entropies.

% Consider placing version information if you circulate multiple drafts
%\vfill
%\begin{center}
%\begin{sf}
%\fbox{Revision: \today}
%\end{sf}
%\end{center}
