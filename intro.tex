%% The following is a directive for TeXShop to indicate the main file
%%!TEX root = diss.tex

\chapter{Introduction}
\label{ch:Introduction}

%\begin{epigraph}
%    \emph{If I have seen farther it is by standing on the shoulders of
%    Giants.} ---~Sir Isaac Newton (1855)
%\end{epigraph}

In the past few decades, significant advances in the field of quantum transport have yielded a large number of interesting quantum systems and effects including Majorana bound states~\cite{frolovmajorana}, the 2-channel Kondo effect~\cite{goldhaber-2ck}, and the $\nu = 5/2$ fractional quantum hall state~\cite{Eisenstein5_2}. All of these systems have been well characterized using traditional transport techniques. However, if we were able to measure the entropy of mesoscopic quantum systems like these, we would be able to more clearly distinguish them from trivial states, and perhaps detect deviations from theory in ways which traditional transport measurements do not allow. Of particular interest is the Majorana bound state whose characteristics make it especially well suited to the field of quantum computing~\cite{simon, kitaev}, but whose transport signature is suspiciously close to that of the much less interesting (and less useful) Andreev bound state~\cite{frolov_mirage}. It has been proposed that the entropy of such a Majorana bound state would significantly differ from that of an Andreev bound state~\cite{majorana_fractional}. However, in the past, entropy measurements of systems like these were never possible because of limitations of techniques which rely on heat capacity and other macroscopic quantities which can be experimentally difficult to measure for localized quantum states.

A few years ago, Hartman et al.~\cite{nikentropy} showed that it is possible to measure the entropy a single \spinh particle in a quantum dot, opening the possibility of introducing entropy as a new technique for characterization of more interesting mesoscopic quantum systems, like those mentioned above. To complete their measurement, Hartman et al. measured electronic occupation in a few-electron quantum dot at varying temperatures, relating this quantity to entropy following a Maxwell relation. Recently, schemes similar to the one realized in this experiment have been used to measure entropy in magic angle (twisted bilayer) graphene~\cite{afy_entropy, pablo_entropy}. Both measurements have shown remarkable evidence of an effective ``freezing" of electrons as temperature is increased -- analogous to the Pomeranchuk effect in $^3$He~\cite{pomeranchuk1950theory}. The results of these recent experiments further illustrate the value in measurements of entropy in quantum materials to provide insights into the electronic states of the system. 

%How to differentiate from Pablo/Shahal Ilani's measurements? I think that the fact they are using quantum capacitance to do this measurement right? To what degree is the data we have showing that "entropy can be measured of the entire thermodynamic system using just a single component that is capacitively coupled" independent from PJH/Ilani's measurements?

In this thesis, we present data to show that the measurement protocol using a single few-electron quantum dot introduced by Hartman et al. to measure the entropy of a single \spinh can be extended to measure the entropy of an additional capacitively coupled quantum system. We provide evidence that, when occupation of a single few-electron quantum dot affects the degeneracy of a larger thermodynamic system, the change in entropy of the entire system can be measured while only measuring the occupation of the single quantum dot. 
\endinput