%% The following is a directive for TeXShop to indicate the main file
%%!TEX root = diss.tex

\chapter{Introduction}
\label{ch:Introduction}

\begin{epigraph}
    \emph{If I have seen farther it is by standing on the shoulders of
    Giants.} ---~Sir Isaac Newton (1855)
\end{epigraph}

In the past few decades, significant advances in the field of quantum transport have yielded a large number of interesting quantum systems and effects including Majorana bound states~\cite{frolovmajorana}, the 2-channel Kondo effect~\cite{goldhaber-2ck}, and the $\nu = 5/2$ fractional quantum hall state~\cite{Eisenstein5_2}. All of these systems have been characterized using traditional transport techniques. However, if we were able to measure the entropy of mesoscopic quantum systems like these, we would be able to more clearly distinguish them from trivial states, and perhaps detect deviations from theory in ways which traditional transport measurements do not allow. Of particular interest is the Majorana bound state whose characteristics make it especially well suited to the field of quantum computing~\cite{simon, kitaev}, but whose transport signature is suspiciously close to that of the much less interesting (and less useful) Andreev bound state~\cite{frolov_mirage}. It has been proposed that the entropy of such a Majorana bound state would significantly differ from that of an Andreev bound state~\cite{majorana_fractional}. However, in the past, entropy measurements of systems like these were never possible because of limitations of old techniques which rely on heat capacity and other macroscopic quantities.

A few years ago, Hartman et al.~\cite{nikentropy} showed that it is possible to measure the entropy a single \spinh particle in a quantum dot, opening the possibility of introducing entropy as a new technique for characterization of more interesting mesoscopic quantum systems, like those mentioned above.  However, while measuring the entropy of one of these systems remains very interesting, it remains to be shown if the protocol that Hartman et al. used to measure the entropy of a single \spinh particle can be extended into a regime where the the quantum dot is capacitively coupled to an external system. Here, we propose a mesoscopic circuit to investigate the effects of capacitively coupling an external quantum system to the measurement scheme of Hartman et al. as a proof that the technique can be extended to the measurement of more interesting quantum systems.
\endinput

Any text after an \endinput is ignored.
You could put scraps here or things in progress.
